This section introduces the reader to three popular feature effect methods; PDPlots, MPlots and ALE. We also design a toy example illustrating that ALE overpasses the weaknesses of PDPlots and MPlots.

\paragraph*{Notation.} We refer to random variables (r.v.) using
uppercase and calligraphic font \( \mathcal{X}\), whereas to simple
variables with plain lowercase \( x \). Bold \( \xb \) denotes a
vector variable, \( \mathcal{X}_s \) the r.v. of the feature of
interest and \( \Xcb \) the rest of the features so that
\( \Xb = (\Xs, \Xcb ) \) represents the input space. The black-box
function is notated as \( f \) and the feature effect of the \(s\)-th
feature as \(f_{\mathtt{<method>}}(x_s)\), where \(\mathtt{<method>}\)
is the name of the feature effect method.\footnote{An extensive list
  of all symbols used in the paper is provided in the Appendix.}

PDPlots formulate the feature effect of the \(s\)-th attribute as an expectation over the marginal distribution \(\mathcal{X}_c\), i.e., \(f_{\mathtt{PDP}}(x_s) = \mathbb{\E}_{\mathcal{X}_c}[f(x_s,\mathcal{X}_c)]\). In contrast, MPlots formulate it as an expectation over the conditional \(\mathcal{X}_c|\mathcal{X}_s\), i.e., \(f_{\mathtt{MP}}(x_s) = \E_{\mathcal{X}_c|\mathcal{X}_s = x_s}[f(x_s, \mathcal{X}_c)]\).

ALE computes the global effect at \(x_s\) as an integration of local effects. The local effects are measured as the expected change on the output \( \frac{\partial f(x_s, \mathcal{X}_c)}{\partial x_s} \) over the conditional distribution \( \Xcb|\mathcal{X}_s\). The formula that defines ALE is presented below:

% \begin{equation}\label{eq:example-1-pdp-definition} % f_{\mathtt{PDP}}(x_s) = \mathbb{\E}_{\mathcal{X}_c}[f(x_s,\mathcal{X}_c)] % = \int_{\xc} p(\xc)f(x_s, \xc) \partial \xc
% \end{equation} % %
% \begin{equation} \label{eq:example-1-mplots-definition} % f_{\mathtt{MP}}(x_s) = \E_{\mathcal{X}_c|\mathcal{X}_s = x_s}[f(x_s, \mathcal{X}_c)] % = \int_{\xc} p(\xc|x_s) f(x_s, \xc) \partial \xc
% \end{equation}

\begin{gather}
  \label{eq:ALE} f_{\mathtt{ALE}}(x_s) = c + \int_{-\infty}^{x_s} \mathbb{E}_{\Xcb|\mathcal{X}_s=z}\left[\frac{\partial f(z, \mathcal{X}_c)}{\partial z}\right] \partial z % = \int_{-\infty}^{x} \int_{\xc} p(\xc|x_s=z)f_s(\xb) \partial \xc \partial z
\end{gather}
%
The constant \(c\) is used for centering the ALE plot. When we refer to the normalized ALE plot we use \(f_{\mathtt{ALE}}\), whereas for the unnormalized we use \(\Tilde{f}_{\mathtt{ALE}}\).

For illustrating the differences between the methods, we provide a toy example. We select a bivariate black-box function \(f\) with correlated features; the first feature \( x_1 \) follows a uniform distribution \( x_1 \sim \mathcal{U}(0,1)\) and the second feature gets the value of \(x_1\) in a deterministic way, i.e., \( x_2 = x_1 \). The black-box function is the following piece-wise linear mapping:
%
\begin{equation} \label{eq:example-1-mapping} f(x_1, x_2) =
  \begin{cases} 1 - x_1 - x_2 & x_1 + x_2 \leq 1 \\ 0 & \text{otherwise}
  \end{cases}
\end{equation}
\noindent

Due to the piece-wise linear form, it is easy to isolate the effect of \(x_1\); In the linear region, i.e., \(0 \leq x_1 \leq 0.5\), it contributes \(-x_1\) and in the constant region, i.e., \(0.5 < x_1 \leq 1 \), the output does not depend on \(x_1\). The closed-form solution for each method is presented below:~\footnote{Detailed derivations can be found in the hepling material.}~\footnote{ Due to symmetry, for each method, the effect for \(x_2\) is the same with the effect of \(x_1\)}
%
\begin{equation}\label{eq:example-1-pdp} f_{\mathtt{PDP}}(x_1) = \mathbb{\E}_{\mathcal{X}_2}[f(x_1,\mathcal{X}_2)] = \frac{{(1-x_1)}^2}{2}, \: \forall x_1 \in [0,1]
\end{equation}
%
\begin{equation} \label{eq:example-1-mplots}
  \begin{split} f_{\mathtt{MP}}(x_1) &= \E_{\mathcal{X}_2|\mathcal{X}_1 = x_1}[f(x_1, \mathcal{X}_2)] = \begin{cases} 1 - 2x_1 & x_1 \leq 0.5 \\ 0 &\text{otherwise}
    \end{cases}
  \end{split}
\end{equation}
%
\begin{align}\label{eq:example-1-ale}
  \begin{split} f_{\mathtt{ALE}}(x_1) &= c + \int_{z_0}^{x_1} \E_{\mathcal{X}_2|\mathcal{X}_1=z} [\frac{\partial f(z, \mathcal{X}_2)}{\partial z}] \partial z =
     \begin{cases} c - x_1 & 0 \leq x_1 \leq 0.5\\ c - 0.5 & 0.5 \leq x_1 \leq 1
    \end{cases}
  \end{split}
\end{align}

The effect computed in Eqs.~\eqref{eq:example-1-pdp},~\eqref{eq:example-1-mplots} helps us understand that PDPlots and MPlots provide misleading results in cases of correlated features. PDPlots integrate over unrealistic instances due to the use of the marginal distribution \( p(\mathcal{X}_1) \). Therefore, they incorrectly result in a quadratic effect in the region \(x_1 \in [0, 1]\). MPlots resolve this issue using the conditional distribution \( \mathcal{X}_2|\mathcal{X}_1 \) but suffer from computing combined effects. In the linear subregion, the effect is overestimated as \( -2x_1 \) which is the combined effect of both \( x_1 \) and \( x_2 \). As emerges from Eq.~\eqref{eq:example-1-ale}, ALE resolves both issues and provides the correct effect.

In real scenarios, we cannot obtain a solution directly from Eq.~\eqref{eq:ALE}. Therefore,~\cite{Apley2020} proposed a solution by splitting the \(x_s\) axis into bins, computing the local effects inside each bin with a Monte Carlo approximation, and, finally, averaging the bin effects. As we discuss extensively in Sections~\ref{sec:4-2-computational} and~\ref{sec:4-3-robustness}, this approximation does not scale well to high-dimensional datasets and is vulnerable to OOD sampling (b). The following section presents DALE, a novel approximation for resolving both issues.
