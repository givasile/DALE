Let a finite dataset of samples \(\mathcal{S}\), drawn independently
and indentically distributed (i.i.d) from the data generating
distribution of \(\mathcal{X}\). DALE computes the accumulated local
effect (Eq.~\eqref{eq:ALE}), using the approximation in
(Eq.~\eqref{eq:DALE}). The expected value of the approximation across
different datasets is

\begin{equation}
  \mathbb{E}_{\mathcal{S}}[f_{\mathtt{DALE}}(x)] =
  \Delta x\sum_{k=1}^{k_x}\mathbb{E}_{\mathcal{S}}[\frac{1}{|\mathcal{S}_k|}\sum_{i:\xb^i \in
      \mathcal{S}_k} f_s(\xb^i)]
  \label{eq:bias_dale}
\end{equation}

\noindent
Notice also that for the values of \(x\) at the end of bin \(k_x\),
Eq.~\eqref{eq:ALE} can be rewritten as (after ommiting the constant
\(c\))
\begin{equation}
  f_{\mathtt{ALE}}(x) = \sum_{k = 1}^{k_x}\int_{x_{k-1}}^{x_k}
    \mathbb{E}_{\Xcb|\mathcal{X}_s=z}[f_s(\xb)] \partial z
    \label{eq:bias_ale_1}
\end{equation}
where \(x_0=x_{s, min}\) and \(x_i\), \(i=1, \dotsc, k_x\) are the bin limits.

\noindent
If we assume that each bin is sufficiently small such that \(f_s(\xb)\) does
not depend on \(x_s\) (i.e., \(f(x)\) is linear wrt \(x_s\)) within the bin, then
Eq. \eqref{eq:bias_ale_1} becomes
\begin{equation}
  f_{\mathtt{ALE}}(x) = \sum_{k = 1}^{k_x}\mathbb{E}_{\Xcb|\mathcal{X}_s \in
    \mathcal{S}_k}[f_s(\xb)]\int_{x_{k-1}}^{x_k} \partial z 
  = \Delta x\sum_{k=1}^{k_x}\mathbb{E}_{\mathcal{X} \in \mathcal{S}_k}[f_s(\xb)]
    \label{eq:bias_ale_2}
\end{equation}

\noindent
From Eqs. \eqref{eq:bias_dale} and \eqref{eq:bias_ale_2} we have
\begin{multline}
    \mathbb{E}_{\mathcal{S}}[f_{\mathtt{DALE}}]  - f_{\mathtt{ALE}}(x) =
    \Delta x\sum_{k=1}^{k_x}\mathbb{E}_{\mathcal{S}}[\frac{1}{|\mathcal{S}_k|}\sum_{k:\xb^k \in
        \mathcal{S}_k} f_s(\xb^k)] - \\
    \Delta x\sum_{k=1}^{k_x}\mathbb{E}_{\mathcal{X}\in \mathcal{S}_k}[f_s(\xb)] = 
  \Delta x\sum_{k=1}^{k_x}\left(\mathbb{E}_{\mathcal{S}}[\hat{\mu}_k^s] - \mu_k^s\right) = 0
\end{multline}
since the expected value of the sample mean is an unbiased estimator
of \(\mu_k^s\). As a result, under the condition of linearity wrt
\(x_s\) within the bin, DALE is an unbiased estimator of the feature
effect. If this assumption is violated (e.g., large bin size or highly
nonlinear function), then this approach may introduce bias. The
variance of the estimator is given\footnote{We show that in the
  supporting material.} by
\( \mathrm{Var}[\hat{\mu}_k^s] =
\dfrac{(\sigma_k^s)^2}{|\mathcal{S}_k|} \), where \((\sigma_k^s)^2\)
is the variance of \(f_s\) within the bin. Furthermore, since the
samples \(\xb^i\) are independent, \(\hat{\mu}_k^s\) for
\(k=1,\dotsc,k_x\) are also independent. The variance of the
estimation can then be approximated as
%
\begin{equation}
  \mathrm{Var}[f_{\mathtt{DALE}}(x)] = (\Delta x)^2\sum_k^{k_x} \mathrm{Var} [\hat{\mu}_k^s] 
  = (\Delta x)^2 \sum_k^{k_x}  \dfrac{(\sigma_k^s)^2}{|\mathcal{S}_k|} \approx
  (\Delta x)^2 \sum_k^{k_x}  \dfrac{(\hat{\sigma}_k^s)^2}{|\mathcal{S}_k|}
  \label{eq:DALE-var}
\end{equation}
%
where \((\hat{\sigma}_k^s)^2\) is the sample variance within bin
\(k\). Equation~\eqref{eq:DALE-var} allows the calculation of the
standard error for the DALE approximation.
