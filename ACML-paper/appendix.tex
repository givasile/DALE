\documentclass[wcp]{jmlr}

%%%% ADDITIONAL
\usepackage[T1]{fontenc}
% givasile packages
\usepackage{bbm}
\usepackage{multirow}
\usepackage{xfrac}
\usepackage[T1]{fontenc}

\usepackage{enumitem}
\usepackage{epstopdf}

\usepackage{bm}
\usepackage{csquotes}
\usepackage{algorithm}
\usepackage{algorithmic}

\newcommand{\xc}{\mathbf{x}_c}
\newcommand{\Xc}{\mathcal{X}_c}
\newcommand{\Xcb}{\mathcal{X}_c}
\newcommand{\Xs}{\mathcal{X}_s}
\newcommand{\Xb}{\mathcal{X}}
\newcommand{\xci}{\mathbf{x}^i_{\mathbf{c}}}
\newcommand{\xb}{\mathbf{x}}
\newcommand{\R}{\mathbb{R}}
\newcommand{\E}{\mathbb{E}}
\newcommand{\Jac}{\mathbf{J}}
\newcommand{\N}{\mathbb{N}}
\newcommand{\B}{\mathbb{B}}

\usepackage{microtype}

\usepackage{tikz}
\usetikzlibrary{matrix,positioning,arrows.meta,arrows,fit,backgrounds,decorations.pathreplacing}

\tikzset{
  mymat/.style={
    matrix of math nodes,
    text height=2.5ex,
    text depth=0.75ex,
    text width=6.00ex,
    align=center,
    column sep=-\pgflinewidth,
    nodes={minimum height=5.0ex}
  },
  mymats/.style={
    mymat,
    nodes={draw,fill=#1}
  },
  mymat2/.style={
    matrix of math nodes,
    text height=1.0ex,
    text depth=0.0ex,
    minimum width=5ex,
%     text width=7.00ex,
    align=center,
    column sep=-\pgflinewidth
  },
}

\usetikzlibrary{shapes.geometric, arrows, backgrounds, scopes}
\usepackage{pgfplots}
\pgfplotsset{width=6.75cm, compat=newest}
\usepackage[utf8]{inputenc}
\DeclareUnicodeCharacter{2212}{−}
\usepgfplotslibrary{groupplots,dateplot}
\usetikzlibrary{patterns,shapes.arrows}

\begin{document}

\section{Notation List}
\label{sec:not-list}
%
\begin{itemize}
\item \( s \), index of the feature of interest
\item \( \mathcal{X}_s \), feature of interest as a r.v.
\item \( \mathcal{X}_c = (\mathcal{X}_{/s}, )\), the rest of the features in as a r.v.
\item \( \mathcal{X} = (\mathcal{X}_s, \mathcal{X}_c) = (\mathcal{X}_1, \cdots, \mathcal{X}_s, \cdots, \mathcal{X}_D) \), all input features as r.v.
\item \( x_s \), feature of interest
\item \( \xc \), the rest of the features
\item \( \xb = (x_s, \xc) = (x_1, \cdots, x_s, \cdots, x_D)\), all the input features
\item \( \mathbf{X} \), design matrix/training set
\item \( f(\cdot) : \R^D \rightarrow \R \), black box function
\item \( f_s(\xb) = \frac{\partial f(x_s, \xc)}{\partial x_s} \), the partial derivative of the \( s \)-th feature
\item \( D \), dimensionality of the input
\item \( N \), number of training examples
\item \( \xb^i \), \(i\)-th training example
\item \( x^i_s \), \(s\)-th feature of the i-th training example
\item \( \xci \), the rest of the features of the i-th training example
\item \( f_{\mathtt{ALE}}(x_s) : \R \rightarrow \R\), ALE definition for the \(s\)-th feature
\item \( \hat{f}_{\mathtt{DALE}}(x_s) : \R \rightarrow \R\), DALE approximation for the \(s\)-th feature
\item \( \hat{f}_{\mathtt{ALE}}(x_s) : \R \rightarrow \R\), ALE approximation for the \(s\)-th feature
\item \( z_{k-1}, z_k\), the left and right limit of the \( k\)-th bin
\item \( \mathcal{S}_k = \{ \xb^i : x^i_s \in [z_{k-1}, z_k) \}\), the set of training points that belong to the \( k\)-th bin
\item \( k_x \) the index of the bin that \( x \) belongs to
\item \( \hat{\mu}_k^s\), \(\mathtt{DALE}\) approximation of the mean value inside a bin, equals \( \frac{1}{|\mathcal{S}_k|} \sum_{i: x^i\in \mathcal{S}_k} f_s(\xb^i) \)
\item \( (\hat{\sigma}_k^s)^2\), \(\mathtt{DALE}\) approximation of the variance inside a bin, equals \( \frac{1}{|\mathcal{S}_k|-1} \sum_{i: x^i\in \mathcal{S}_k} (f_s(\xb^i) - \hat{\mu}_k^s)^2 \)

\end{itemize}

\section{Derivation of equations in the Background section}

In this section, we present the derivations for obtaining the feature
effect at the Background.


\subsubsection*{Example Definition.}The black-box function and the
generating distribution are:

\begin{equation}
  \label{eq:black-box}
  f(x_1, x_2) =
  \begin{cases}
    1 - x_1 - x_2 \: \: \:  ,\text{if} \: x_1 + x_2  \leq 1 \\
    0 \quad \quad \quad \quad \quad \:, \text{otherwise}
  \end{cases}
\end{equation}

\begin{equation}
  \label{eq:generative}
  p(\mathcal{X}_1 = x_1, \mathcal{X}_2=x_2) =
  \begin{cases}
    1 & x_1 \in [0,1], x_2=x_1 \\
    0 & \text{otherwise}
  \end{cases}
\end{equation}

\begin{equation}
  \label{eq:marginal}
  p(\mathcal{X}_1 = x_1) =
  \begin{cases}
    1 & 0 \leq x_1 \leq 1 \\
    0 & \text{otherwise}
  \end{cases}
\end{equation}

\begin{equation}
  \label{eq:marginal}
  p(\mathcal{X}_2 = x_2) =
  \begin{cases}
    1 & 0 \leq x_2 \leq 1 \\
    0 & \text{otherwise}
  \end{cases}
\end{equation}

\begin{equation}
  \label{eq:marginal}
  p(\mathcal{X}_2 = x_2|\mathcal{X}_1 = x_1) = \delta(x_2-x_1)
\end{equation}


\subsubsection*{PDPlots.}

The feature effect computed by PDP plots is:

\begin{equation}
  \label{eq:example-1-pdp}
  \begin{split}
    f_{\mathtt{PDP}}(x_1) &= \\
    & = \mathbb{\E}_{\mathcal{X}_2}[f(x_1,\mathcal{X}_2)] \\
    & = \int_{x_2} f(x_1, x_2) p(x_2) \partial x_2 \\
    & = \int_{0}^{1-x_1} (1 - x_1 - x_2) \partial x_2 + \int_{1-x_1}^1 0 \partial x_2 \\
    & = \int_{0}^{1-x_1} 1 \partial x_2 + \int_{0}^{1-x_1} -x_1 \partial x_2 + \int_{0}^{1-x_1} -x_2 \partial x_2 \\
    & = (1 - x_1) -x_1(1-x_1) - \frac{{(1-x_1)}^2}{2} \\
    & = (1 - x_1)^2 - \frac{{(1-x_1)}^2}{2} \\
    & = \frac{{(1-x_1)}^2}{2}
  \end{split}
\end{equation}

%
Due to symmetry:

\begin{equation}
y = f_{\mathtt{PDP}}(x_2) = \frac{{(1-x_2)}^2}{2}
\end{equation}

\subsubsection*{MPlots.}

The feature effect computed by PDP plots is:

\begin{equation}
  \label{eq:example-1-MPlots}
  \begin{split}
    f_{\mathtt{MP}}(x_1) &= \\
    & = \mathbb{\E}_{\mathcal{X}_2|\mathcal{X}_1=x_1}[f(x_1,\mathcal{X}_2)] \\
    & = \int_{x_2} f(x_1,x_2) p(x_2|x_1) \partial x_2 \\
    & =   f(x_1, x_1) = \\
  & = \begin{cases}
    1 - 2x_1, & x_1 \leq 0.5 \\
    0, & \text{otherwise}
\end{cases}
  \end{split}
\end{equation}
%
Due to symmetry:

\begin{equation}
  y = f_{\mathtt{MP}}(x_2) =
  \begin{cases}
    1 - 2x_2 & x_2 \leq 0.5 \\
    0, &\text{otherwise}
  \end{cases}
\end{equation}

\subsubsection*{ALE}

The feature effect computed by ALE is:

\begin{equation}
  \label{eq:example-1-ale}
  \begin{split}
    f_{\mathtt{ALE}}(x_1) &= \\
    & = \int_{z_0}^{x_1} \mathbb{E}_{\mathcal{X}_2|\mathcal{X}_1=z} \left [\frac{\partial f}{\partial z}(z, \mathcal{X}_2) \right ] \partial z \\
    & = \int_{z_0}^{x_1} \int_{x_2} \frac{\partial f}{\partial z}(z,x_2) p(x_2|z)  \partial x_2 \partial z = \\
    & = \int_{z_0}^{x_1} \frac{\partial f}{\partial z}(z,z) \partial z = \\
    & = \begin{cases}
      \int_{z_0}^{x_1} -1 \partial z & x_1 \leq 0.5 \\
      \int_{z_0}^{0.5} -1 \partial z + \int_{.5}^{x_1} 0 \partial z & x_1 > 0.5
    \end{cases} \\
    & = \begin{cases}
      -x_1 & x_1 \leq 0.5 \\
      -0.5 & x_1 > 0.5
    \end{cases}
  \end{split}
\end{equation}

The normalization constant is:

\begin{equation}
  \label{eq:constant}
  \begin{split}
    c & = - \mathbb{E}[\hat{f}_{ALE}(x_1)] \\
    & = - \int_{-\infty}^{\infty} \hat{f}_{ALE}(x_1) \\
    & = - \int_{0}^{0.5} - z \partial z - \int_{0.5}^{1} -0.5 \partial z \\
    & = \frac{0.25}{2} + 0.25 = 0.375
  \end{split}
\end{equation}

Therefore, the normalized feature effect is:

\begin{gather}
y = f_{\mathtt{ALE}}(x_1) =
\begin{cases}
0.375 - x_1 & 0 \leq x_1 \leq 0.5\\
- 0.125 &  0.5 < x_1 \leq 1
\end{cases}
\end{gather}

Due to symmetry:

\begin{gather}
y = f_{ALE}(x_2) =
\begin{cases}
0.375 - x_2 & 0 \leq x_2 \leq 0.5\\
- 0.125 &  0.5 < x_2 \leq 1
\end{cases}
\end{gather}

\section{First-order and Second-order DALE approximation}

In the main part of the paper, we presented the first order ALE approximation as

\begin{align}
  f_{\mathtt{DALE}}(x_s) = \Delta x \sum_{k=1}^{k_x} \frac{1}{|\mathcal{S}_k|}
  \sum_{i:\xb^i \in \mathcal{S}_k} [f_s(\xb^i)]
\end{align}
%
For keeping the equation compact, we ommit a small detail about the
manipulation of the last bin. In reality, we take complete
\( \Delta x \) steps until the \( k_x - 1 \) bin, i.e. the one that
prepends the bin where \( x \) lies in. In the last bin, instead of a
complete \( \Delta x \) step, we move only until the position \( x
\). Therefore, the exact first-order DALE approximation is

\begin{equation}
  \begin{split}
  f_{\mathtt{DALE}}(x_s) &= \Delta x \sum_{k=1}^{k_x-1} \frac{1}{|\mathcal{S}_k|}
  \sum_{i:\xb^i \in \mathcal{S}_k} [f_s(\xb^i)]\\
  & + (x-z_{(k_x-1)}) \frac{1}{|\mathcal{S}_{k_x}|} \sum_{i:\xb^i \in
    \mathcal{S}_{k_x}} [f_s(\xb^i)]
  \end{split}
  \label{eq:DALE_first_order_complete}
\end{equation}

\noindent
Following a similar line of thought we define the complete second-order DALE approximation as

\begin{multline}
  f_{\mathtt{DALE}}(x_l, x_m) = \Delta x_l\sum_{p=1}^{p_x-1} \Delta x_m\sum_{q=1}^{q_x-1} \frac{1}{|\mathcal{S}_{k,q}|} \sum_{i:\xb^i \in \mathcal{S}_{k,q}}f_{l,m}(\xb^i)\\
  + (x_l-z_{(p_x-1)})(x_m-z_{(q_x-1)}) \frac{1}{|\mathcal{S}_{p_x,q_x}|} \sum_{i:\xb^i \in \mathcal{S}_{p_x,q_x}}f_{l,m}(\xb^i)
 \label{eq:DALE_second_order_complete}
\end{multline}

\section{Second-order ALE definition}

The second-order ALE plot definintion is

\begin{equation}
  f_{\mathtt{ALE}}(x_l, x_m) = c + \int_{x_{l, min}}^{x_l} \int_{x_{m, min}}^{x_m} \mathbb{E}_{\Xc|X_l=z_l,
      X_m=z_m}[f_{l,m}(\xb)] \partial z_l \partial z_m
  \label{eq:ALE2}
\end{equation}

\noindent

where
\( f_{l,m}(\xb) = \dfrac{\partial^2f(x)}{\partial x_l \partial x_m} \).

\section{DALE variance inside each bin}

In this section, we show that the variance of the local effect
estimation inside a bin, i.e. \(\mathrm{Var}[\hat{\mu}_k^s]\) equals
with \(\frac{(\sigma_k^s)^2}{|\mathcal{S}_k|}\), where
\((\sigma_k^s)^2 = \mathrm{Var}[f_s(\mathbf{x})]\).

\begin{equation}
  \begin{split}
  \mathrm{Var}[\hat{\mu}_k^s] &= \mathrm{Var} [\frac{1}{|\mathcal{S}_k|} \sum_{i: x^i\in \mathcal{S}_k} f_s(\xb^i)] \\
                              &= \frac{1}{|\mathcal{S}_k|^2} \sum_{i: x^i\in \mathcal{S}_k} \mathrm{Var}[f_s(\xb^i)] \\
                              &= \frac{|\mathcal{S}_k|}{|\mathcal{S}_k|^2} \mathrm{Var}[f_s(\xb)] \\
  &= \frac{(\sigma_k^s)^2}{|\mathcal{S}_k|}  \\
  \end{split}
\end{equation}

% \section{DALE variance}
% \label{sec:dale-variance}

% In this section we show that the variance of the approximation
% \( \mathrm{Var}[f^s_{\mathtt{DALE}}(x)] \) equals with
% \( (\Delta x)^2 \sum_k^{k_x} \dfrac{(\sigma_k^s)^2}{|\mathcal{S}_k|}
% \) and can be approximated by
% \((\Delta x)^2 \sum_k^{k_x}
% \dfrac{(\hat{\sigma}_k^s)^2}{|\mathcal{S}_k|}\). We observe that the
% variance at the \( k \)-th bin is a sum of all the previous bins.

% \begin{equation}
%   \begin{split}
%   \mathrm{Var}[f^s_{\mathtt{DALE}}(x)] &= \mathrm{Var} [\Delta x \sum_k^{k_x} \hat{\mu}_k^s] \\
%                                       &= (\Delta x)^2 \sum_k^{k_x} \mathrm{Var}[\hat{\mu}_k^s] \\
%                                       &= (\Delta x)^2 \sum_k^{k_x} \dfrac{(\sigma_k^s)^2}{|\mathcal{S}_k|}\\
%                                       &\approx (\Delta x)^2 \sum_k^{k_x}  \dfrac{(\hat{\sigma}_k^s)^2}{|\mathcal{S}_k|}
%   \end{split}
% \end{equation}

% \noindent
% Therefore, the standard deviation is

% \begin{equation}
%   \mathrm{Std}[f^s_{\mathtt{DALE}}(x)] = \sqrt{\mathrm{Var}[f^s_{\mathtt{DALE}}(x)]} = \Delta x \sqrt{\sum_k^{k_x}  \dfrac{(\hat{\sigma}_k^s)^2}{|\mathcal{S}_k|}}
% \end{equation}



\end{document}