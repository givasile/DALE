%%%% ijcai22.tex

%\typeout{IJCAI--22 Instructions for Authors}

% These are the instructions for authors for IJCAI-22.

\documentclass{article}
\pdfpagewidth=8.5in
\pdfpageheight=11in
% The file ijcai22.sty is NOT the same as previous years'
\usepackage{ijcai22}

% Use the postscript times font!
\usepackage{times}
\usepackage{soul}
\usepackage{url}
\usepackage[hidelinks]{hyperref}
\usepackage[utf8]{inputenc}
\usepackage[small]{caption}
\usepackage{graphicx}
\usepackage{amsmath}
\usepackage{amsthm}
\usepackage{booktabs}
\usepackage{algorithm}
\usepackage{algorithmic}
\urlstyle{same}


% customly used packages
% \usepackage{mathfont}
\usepackage{amssymb}
\usepackage{tikz}
\usetikzlibrary{shapes.geometric, arrows, backgrounds, scopes}
\usepackage{pgfplots}
\pgfplotsset{width=6.75cm, compat=newest}
\usepackage[utf8]{inputenc}
\DeclareUnicodeCharacter{2212}{−}
\usepgfplotslibrary{groupplots,dateplot}
\usetikzlibrary{patterns,shapes.arrows}

% the following package is optional:
%\usepackage{latexsym}
\newcommand\todo[1]{\textcolor{red}{Givasile and/or Diou:} \textcolor{red}{#1}}
\newcommand\tododiou[1]{\textcolor{blue}{Diou:} \textcolor{blue}{#1}}
\newcommand\todogiv[1]{\textcolor{purple}{givasile:} \textcolor{purple}{#1}}

\newcommand{\xc}{\mathbf{x}_c}
\newcommand{\Xc}{\mathbf{\mathcal{X}}_c}
\newcommand{\xci}{\mathbf{x}^i_{\mathbf{c}}}
\newcommand{\xb}{\mathbf{x}}
\newcommand{\R}{\mathbb{R}}
\newcommand{\E}{\mathbb{E}}
\newcommand{\Jac}{\mathbf{J}}
% \newcommand{\ale}{f_{ALE}}
% \newcommand{\alea}{\hat{f}_{ALE}}
% \newcommand{\myale}{\tilde{f}_{ALE}}

\usepackage{xfrac}
\usepackage{mleftright}
\usepackage{xparse}
\NewDocumentCommand{\evalat}{sO{\big}mm}{%
  \IfBooleanTF{#1}
   {\mleft. #3 \mright|_{#4}}
   {#3#2|_{#4}}%
}
% See https://www.overleaf.com/learn/latex/theorems_and_proofs
% for a nice explanation of how to define new theorems, but keep
% in mind that the amsthm package is already included in this
% template and that you must *not* alter the styling.
\newtheorem{example}{Example}
\newtheorem{theorem}{Theorem}

% Following comment is from ijcai97-submit.tex:
% The preparation of these files was supported by Schlumberger Palo Alto
% Research, AT\&T Bell Laboratories, and Morgan Kaufmann Publishers.
% Shirley Jowell, of Morgan Kaufmann Publishers, and Peter F.
% Patel-Schneider, of AT\&T Bell Laboratories collaborated on their
% preparation.

% These instructions can be modified and used in other conferences as long
% as credit to the authors and supporting agencies is retained, this notice
% is not changed, and further modification or reuse is not restricted.
% Neither Shirley Jowell nor Peter F. Patel-Schneider can be listed as
% contacts for providing assistance without their prior permission.

% To use for other conferences, change references to files and the
% conference appropriate and use other authors, contacts, publishers, and
% organizations.
% Also change the deadline and address for returning papers and the length and
% page charge instructions.
% Put where the files are available in the appropriate places.

% PDF Info Is REQUIRED.
% Please **do not** include Title and Author information
\pdfinfo{
/TemplateVersion (IJCAI.2022.0)
}

\title{DALE: An efficient and accurate approximation method for ALE plots}

% Multiple author syntax (remove the single-author syntax above and the \iffalse ... \fi here)
% Check the ijcai22-multiauthor.tex file for detailed instructions
%% \author{
%% Vasileios Gkolemis\(^1\)
%% \and
%% Christos Diou\(^2\)\and
%% Theodore Dalamagas \(^{1}\)
%% \affiliations
%% \(^1\)First Affiliation\\
%% \(^2\)Department of Informatics and Telematics, Harokopio University of Athens, Greece
%% \emails
%% \{first, second\}@example.com,
%% cdiou@hua.gr,
%% }

\begin{document}

\maketitle

\begin{abstract}
  Aggregated Local Effect (ALE) is a method for accurate estimation
  and visualization of feature effects, which overcomes fundamental
  failure modes of previously-existed methods, such as Partial
  Dependence Plots (PDP). The \textit{approximation} method of ALE,
  however, faces two crucial weaknesses. Firstly, it does not scale
  well in cases where the input has high dimensionality. Secondly, it
  is vulnerable to out-of-distribution sampling in cases with limited
  training examples. In this paper we propose a novel ALE
  approximation, called Differential Aggregated Local Effects (DALE),
  which overcomes these issues. Our proposal has significant
  computational advantages; Specifically, the computational complexity
  for computing the feature effect for all attributes using DALE is
  similar to what ALE requires for a single attribute. DALE,
  therefore, makes feature effect applicable to high-dimensional
  Machine Learning scenarios with near-zero computational
  overhead. Furthermore, unlike ALE, it calculates the feature effect
  using only the available samples, without creating artificial ones;
  This strategy resolves the weakness of misleading estimations due to
  out-of-distribution sampling. Experiments using both synthetic and
  real datasets demonstrate the value of the proposed approach. Code
  to reproduce these experiments will become publicly available upon
  acceptance.
\end{abstract}

\section{Introduction}
Recently, Machine Learning (ML) models have flourished in critical, high-stakes application domains, such as healthcare and finance. These fields require methods with the ability to explain their predictions, i.e., justify why a specific outcome has emerged. However, several types of accurate and highly non-linear models like Deep Neural Networks do not meet this requirement. Therefore, there is a growing need for explainability methods for interpreting such ``black-box'' models.

Feature effect forms a fundamental category of global explainability methods (i.e. characterizing the model as a whole, not a particular input). The goal of the feature effect is to isolate the average impact of a single feature on the output. This class of methods is attractive due to the simplicity of the explanation that is easily understandable by a non-expert.

There are three popular feature effect methods: (i) Partial Dependence Plots (PDPlots)~\cite{Friedman2001}, (ii) Marginal Plots (MPlots)~\cite{Apley2020} and (iii) Aggregated Local Effects (ALE)~\cite{Apley2020}. PDPlots and MPlots assume that input features are not correlated. When this does not hold, both methods perform poorly; PDPlots quantify the effect by marginalizing over out-of-distribution synthetic instances (OOD), and MPlots attribute aggregated effects on single features. Therefore, both methods perform well only in the case of independent or low-correlated features. ALE is the only feature effect method that succeeds in staying on distribution and isolating feature effects in typical ML scenarios where input features are highly-correlated\footnote{In Section~\ref{sec:3-feature-effect}, we provide a thorough analysis for clarifying the differences between these three approaches.}.

In most cases, it is impossible to compute ALE through its definition since this would require (a) solving a high-dimensional integral, which is infeasible, and (b) evaluating the data generating distribution, which is usually unknown. Therefore, the original ALE paper~\cite{Apley2020} proposed a Monte-Carlo approximation, which faces two weaknesses. First, it becomes computationally inefficient in cases of datasets with numerous high-dimensional instances. Second, it is still vulnerable to OOD sampling in cases of wide bin sizes.

This paper proposes Differential Aggregated Local Effects (DALE), a novel approximation for ALE that resolves both weaknesses. DALE leverages auto-differentiation for computing the derivatives wrt each instance in a single pass. Therefore, it scales well in the case of high-dimensional inputs, large training sets and expensive black-box models. Furthermore, DALE estimates the feature effect using only the examples from the training set, securing that the estimation is not affected by OOD samples.
%
The contributions of this work are:
%
\begin{itemize}
\item We introduce DALE, a novel approximation to efficiently create ALE plots on differentiable black-box models. DALE is more efficient than the traditional ALE approximation, scales much better to high-dimensional datasets, and avoids OOD sampling.
\item We formally prove that DALE is an unbiased estimator of ALE and quantify the standard error of the approximation.
\item We experiment with synthetic and real datasets, showing that DALE: (a) scales in all cases better than ALE, (b) provides a better approximation compared to ALE, especially in cases of wide bin sizes
\end{itemize}


\section{Related Work}
Explainable AI (XAI) is a fast-evolving field with a growing
interest. In recent years, the domain has matured by establishing its
terminology and objectives~\cite{Hoffman2018}. Several surveys have
been published~\cite{BarredoArrieta2020}, % ~\cite{Adadi2018}
classifying
the different approaches and detecting future challenges on the
field~\cite{Molnar2020}.

There are several criterias for grouping XAI methods. A very popular
distinction is between local and global ones. Local interpretability
methods explain why a model made a specific prediction given a
specific input. For example, local surrogates such as
LIME~\cite{Ribeiro2016} train an explainable-by-design model in data
points generated from a local area around the input under
examination. SHAP values~\cite{Lundberg2017} measures the contribution
of each attribute in a specific prediction, formulating a
game-theoretical framework based on Shapley
Values. Counterfactuals~\cite{Wachter2017} search for a data point as
close as possible to the examined input that flips the
prediction. Anchors~\cite{Ribeiro2018} provide a rule, i.e. a set of
attribute values, that is enough to freeze the prediction,
independently of the value of the rest of the attributes.


Global methods, which is the focus of this paper, explain the average
model behaviour. For example, prototypes~\cite{Gurumoorthy2019} search
for a data point that is a characteristic representative of a specific
class. Criticisms~\cite{Kim2016}, search for data points whose class
is ambiguous. Global feature importance methods characterize each
input feature by assigning to it an importance score. Permutation
feature importance~\cite{Fisher2019} measures the change in the
prediction score of a model, after permuting the value of each
feature. Often, apart from knowing that a feature is important, it
also valuable to know the type of the effect on the output
(positive/negative). Feature effect methods take a step further and
quantify the type of a each feature attribute influences the output on
average. There are three popular feature effect techniques Partial
Dependence Plots~\cite{Friedman2001}, Marginal Plots and
ALE~\cite{Apley2020}. Another class of global explanation techniques
measures the interaction~\cite{Friedman2008} between features. Feature
interaction quantifies to what extent the effect of two variables on
the output comes is because of their combination.~\cite{Friedman2008}
proposed a set of appropriate visualizations for such
interactions. The generalization of feature effect and variable
interactions is functional decomposition~\cite{Molnar2021}, that
decomposes the black-box function into a set of simpler ones that may
include more than two features.


\section{Method}

In this section, we present the DALE approximation;
In~\ref{sec:3-1-DALE}, we formulate the expression for the first and
second-order DALE, in~\ref{sec:3-2-computational} we explain its
computational benefits, in~\ref{sec:3-3-robustness} its robustness to
OOD sampling and, finally in~\ref{sec:3-4-std}, we quantify the
standard error of the DALE estimation.

\subsection{Differential Aggregate Local Effect (DALE)}
\label{sec:3-1-DALE}
\input{./chapters/3-1-DALE.tex}

\subsection{Computational Benefit}
\label{sec:3-2-computational}
\input{./chapters/3-2-computational.tex}

\subsection{Robustness to out-of-distribution sampling}
\label{sec:3-3-robustness}
\input{./chapters/3-3-robustness.tex}

\subsection{Bias and variance of the DALE approximation}
\label{sec:3-4-std}
\input{./chapters/3-4-std.tex}

% \subsection{Limitations of DALE}
% \label{sec:3-5-limitations}
% \input{./chapters/3-5-limitations.tex}

\section{Experiments}

This section provides an experimental evaluation of DALE in
synthetic and real datasets. In the first part, we generate artificial
data points using a known generating distribution \(p(\mathcal{X}) \)
and a mapping \( f: \mathbf{X} \rightarrow Y \). The second part tests
our proposed method in a real dataset. In this case, we do not possess
ground truth information; threfore, we discuss the qualitative
differences between the two approximations. Code to reproduce the
experiments of this section will become publicly available upon
acceptance.

\subsection{Artificial Examples}
\input{./chapters/4-1-artificial-experiments.tex}

\subsection{The bike-sharing dataset}
\input{./chapters/4-2-real-datasets.tex}

\section{Conclusion}
\input{./chapters/5-conclusion.tex}

\subsubsection*{Acknowledgements}

Redacted.

%% The file named.bst is a bibliography style file for BibTeX 0.99c
\bibliographystyle{named}
\bibliography{ijcai22}

\end{document}
