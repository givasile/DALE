\documentclass[final]{beamer}
%% Possible paper sizes: a0, a0b, a1, a2, a3, a4.
%% Possible orientations: portrait, landscape
%% Font sizes can be changed using the scale option.
\usepackage[size=a0,orientation=portrait,scale=1.3]{beamerposter}

\usetheme{gemini}
\usecolortheme{seagull}
\useinnertheme{rectangles}

% ====================
% Packages
% ====================

\usepackage[utf8]{inputenc}
\usepackage{graphicx}
\usepackage{bm, amsmath}
% \usepackage[most]{tcolorox}
\usepackage{wrapfig}
\usepackage{booktabs}
\usepackage{tikz}
\usepackage{pgfplots}

\newcommand{\xb}{\mathbf{x}}

% ====================
% Lengths
% ====================

% If you have N columns, choose \sepwidth and \colwidth such that
% (N+1)*\sepwidth + N*\colwidth = \paperwidth
\newlength{\sepwidth}
\newlength{\colwidth}
\setlength{\sepwidth}{0.03\paperwidth}
\setlength{\colwidth}{0.45\paperwidth}

\newcommand{\separatorcolumn}{\begin{column}{\sepwidth}\end{column}}

% ====================
% Logo (optional)
% ====================

% LaTeX logo taken from https://commons.wikimedia.org/wiki/File:LaTeX_logo.svg
% use this to include logos on the left and/or right side of the header:
\logoright{\includegraphics[height=5cm]{logos/athena-278x300.png}}
\logoleft{\includegraphics[height=5cm]{logos/dit-logo.png}}

% ====================
% Footer (optional)
% ====================

\footercontent{
	ACML 2022, Hyderabad, India \hfill
	\insertdate \hfill
	\href{mailto:vgkolemis@athenarc.gr}{\texttt{vgkolemis@athenarc.gr}}
}
% (can be left out to remove footer)

% ====================
% My own customization
% - BibLaTeX
% - Boxes with tcolorbox
% - User-defined commands
% ====================
\input{custom-defs.tex}

%% Reference Sources
\addbibresource{refs.bib}
\renewcommand{\pgfuseimage}[1]{\includegraphics[scale=2.0]{#1}}

\title{DALE: Differential Accumulated Local Effects for accurate and efficient global effect estimation}

\author{Vasilis Gkolemis \inst{1, 2} \and Theodore Dalamagas \inst{1} \and Christos Diou \inst{2}}

\institute[shortinst]{\inst{1} ATHENA Research Center \samelineand \inst{2} Harokopio University of Athens}

\date{January 01, 2025}

\begin{document}
	
\begin{frame}[t]
	
	\begin{columns}[t] \separatorcolumn
		\begin{column}{\colwidth}
      \begin{block}{TL;DR}
        \Large{\textbf{DALE} is a better approximation to ALE}, the
        SotA feature effect method. By better, we mean faster and more
        accurate.

        \large{\textbf{keywords:} eXplainable AI, global, model-agnostic, deep learning}
			\end{block}

			\begin{block}{Motivation}{}
        Feature effect methods are simple and intuitive; they isolate
        the impact of a single feature \(x_s\) in the output
        \(y\). Inspecting the feature effect plot a non-expert can
        easily understand whether a feature has positive/negative on
        the target variable.

        The task is difficult; isolating the effect of a single
        variable is tricky when features are correlated and the
        black-box function has learned complex. ALE is the only method
        that manages. However, ALE estimation, i.e., the approximation
        of ALE from the set of has efficiency and accuracy that we
        address with DALE.
			\end{block}

      %%%%%% DALE vs ALE
      \begin{block}{DALE vs ALE}{}
        \begin{tcolorbox}[ams equation*, title=ALE definition]
          f(x_s) = \int_{x_{s,
              min}}^{x_s}\mathbb{E}_{\bm{x_c}|z} \underbrace{[\frac{\partial
            f}{\partial x_s}(z, \bm{x_c})]}_{\text{point effect}} \partial z
        \end{tcolorbox}
        ALE defines the effect at \(x_s=z\) as the expected change
        (derivative) on the output over the conditional distribution
        \(\bm{x_c}|z\) and the feature effect plot as the integration
        of the expected changes. 

        \begin{tcolorbox}[ams equation*, title=ALE approximation]
          f(x_s) = \sum_k^{k_x} \underbrace{\frac{1}{|\mathcal{S}_k|}
            \sum_{i:\mathbf{x}^i \in \mathcal{S}_k} \underbrace{[f(z_k,
              \bm{x^i_c}) - f(z_{k-1}, \bm{x^i_c})]}_{\text{point
                effect}}}_{\text{bin effect}}
        \end{tcolorbox}


        ALE approximation, i.e. estimating ALE from the training set
        \(\mathcal{D}\), requires partitioning the \(s\)-th axis, i.e.
        \([x_{s,min}, x_{s,max}]\), in \(K\) equisized bins and
        computes the \textit{local} point effects by evaluating the
        bin limits \([f(z_k, \bm{x^i_c}) - f(z_{k-1},
        \bm{x^i_c})]\). This approach is slow and vulnerable to
        misestimatons. First, it is slow as the dimensionality of the
        dataset grows larger. Second, it demands predifing the bin
        limits. Finally, it may create out-of-distribution samples
        when bin size becomes large.

\begin{tcolorbox}[ams equation*, title=DALE approximation]
       f(x_s) = \Delta x \sum_k^{k_x}
\underbrace{\frac{1}{|\mathcal{S}_k|} \sum_{i:\xb^i \in \mathcal{S}_k}
\underbrace{[\frac{\partial f}{\partial x_s}(x_s^i,
\bm{x^i_c})]}_{\text{\alert{point effect}}}}_{\text{bin effect}}
\end{tcolorbox}

DALE approximation addresses these issues. The main difference is the
use of automatic differentiation, instead of evaluating at the bin
limits.

  \end{block}

\end{column}

		\separatorcolumn
		\begin{column}{\colwidth}
			\begin{block}{DALE saves from OOD}

% \begin{wrapfigure}{r}{5.5cm}
% \caption{A wrapped figure going nicely inside the text.}\label{wrap-fig:1}
% \includegraphics[width=5.5cm]{sample}
% \end{wrapfigure} 
% %------------------------------------------
        \begin{tabular}{cl}  
          \begin{tabular}{c}
            \includegraphics[width=.5\textwidth]{./../ACML-presentation/figures/bin_splitting_5_bins.pdf}
          \end{tabular}
          & \begin{tabular}{l}
              \parbox{0.5\linewidth}{%  change the parbox width as appropiate
              Consider the following case; (a) we have limited samples (b) high variance and (c) the black-box function changes abruptply outside the data manifold. For example, \( f(x_1, x_2, x_3) = x_1x_2 + x_1x_3 \: \textcolor{red}{ \pm \: g(x)}\), with \(x_1 \in [0,10]\), \(x_2=x_1 + \epsilon\) and \(x_3 \sim \mathcal{N}(0, \sigma^2)\). The term \(x_1x_3\) makes estimations from limited samples noisy, so we need to grow the bins larger (more \( \frac{\text{points}} { \text{bin}} \)). But as we grow bins, ALE creates OOD samples, 
    }
            \end{tabular}  \\
        \end{tabular}

        \begin{figure}
          \centering
          \includegraphics[width=0.49\textwidth]{./../ACML-presentation/figures/dale_40_bins.pdf}
          \includegraphics[width=0.49\textwidth]{./../ACML-presentation/figures/ale_40_bins.pdf}
        \end{figure}
        
        \begin{figure}
          \centering
          \includegraphics[width=0.49\textwidth]{./../ACML-presentation/figures/dale_5_bins.pdf}
          \includegraphics[width=0.49\textwidth]{./../ACML-presentation/figures/ale_5_bins.pdf}
        \end{figure}
			\end{block}
			
			\begin{block}{DALE is much more efficient}
				
        \begin{figure}[ht]
          \centering
          \includegraphics[width=0.49\textwidth]{./../ACML-paper/images/case-1-plot-1.pdf}
          \includegraphics[width=0.49\textwidth]{./../ACML-paper/images/case-1-plot-2.pdf}
          \caption[Case-1-fig-1]{Light setup; small dataset \((N=10^2\) instances), light \(f\). Heavy setup; big dataset (\(N=10^5\) instances), heavy \(f\)}
        \end{figure}
			\end{block}
		\end{column}
		\separatorcolumn
	\end{columns}

	\begin{columns}[t]\separatorcolumn
	\begin{column}{1.3\colwidth}
	\begin{block}{A block title}
		This poster was made by modifying the Gemini Beamer Poster Theme~\parencite{Athalye2018} and the Beamer \texttt{seagull} Color Theme.
		Some block contents, followed by a diagram, followed by a dummy paragraph.
		
		Lorem ipsum dolor sit amet, consectetur adipiscing elit. Morbi ultricies
		eget libero ac ullamcorper. Integer et euismod ante. Aenean vestibulum
		lobortis augue, ut lobortis turpis rhoncus sed. Proin feugiat nibh a
		lacinia dignissim. Proin scelerisque, risus eget tempor fermentum, ex
		turpis condimentum urna, quis malesuada sapien arcu eu purus.
	\end{block}
	\end{column}
  \separatorcolumn
	\begin{column}{0.7\colwidth}
	\begin{block}{References}
				\printbibliography[heading=none]
	\end{block}
\end{column}
\separatorcolumn
\end{columns}

\end{frame}

\end{document}